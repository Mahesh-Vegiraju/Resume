\documentclass{resume}
\usepackage[hidelinks]{hyperref}
\usepackage[normalem]{ulem}

\usepackage{fontspec} %
\setmainfont{Lucida Grande} %
\renewcommand\familydefault{\sfdefault} 
\usepackage[T1]{fontenc}

\usepackage[left=0.5in, top=0.5in, right=0.5in, bottom=0.3in]{geometry}
\newcommand{\tab}[1]{\hspace{.2667\textwidth}\rlap{#1}}
\newcommand{\itab}[1]{\hspace{0em}\rlap{#1}}
\name{Mahesh Vegiraju}

\address{
 email:mvegiraj@ucsc.edu \\
 in/MaheshVegiraju \\
 github:Mahesh-Vegiraju 
}

\begin{document}

%----------------------------------------------------------------------------------------
%	EDUCATION SECTION
%----------------------------------------------------------------------------------------
\begin{rSection}{\large Education}

{\bf University of California, Santa Cruz} \hfill {\bf{Expected --- June 2023}}
\\ Bachelor of Science Computer Science\hfill {\emph{GPA: 3.5}} 
\\ Minor in Biology 

\\ \textbf{Relevant CS Coursework:} Programming Abstractions in Python, Assembly Language \& Computer Systems, Embedded Systems \& C Programming, Data Structures and Algorithms, Computer Architecture, Introduction to Computer Networks, Computer Systems Design, Foundations of Programming Languages, Introduction to Software Engineering
\\ \textbf{Relevant Biology Coursework:} General Chemistry I, General Chemistry II, General Chemistry III, Cell \& Molecular Biology, Biology: Development \& Physiology, Genetics 
\\ \textbf{Involvements:}  Santa Cruz NeuroTech, Cycling Club, Brooks Lab

\end{rSection}


%----------------------------------------------------------------------------------------
%	EXPERIENCE SECTION
%----------------------------------------------------------------------------------------
\begin{rSection}{\large Experience}

\begin{rSubsection}{Brooks Lab}{\bf{Feburary 2022 - Present}}{Undergraduate Researcher}{Santa Cruz, CA }
\item Working with long read \& short read simulated data to compare statistical data between the two
\item Working with technologies like FLAIR, JAFFAR, AERON, ARIBA and STAR
\end{rSubsection}

\begin{rSubsection}{CalFee Design}{\bf{November 2021 - Feburary 2022}}{Independent Contractor}{Santa Cruz, CA }
\item Worked with motor controllers, hall sensors, buttons and arduinos to create the brains of an ebike system
\item Implemented quality of life features like dynamic motor activation and smooth motor ramp up/down
\end{rSubsection}

\begin{rSubsection}{Acorn Basket Studios}{\bf{June 2019 - August 2019}}{Software Engineering Intern - Game Mechanics}{San Jose, CA }
\item Helped implement object collision \& procedural generation of the game map
\item Worked in the ActionScript framework to make the game accessible to a wider audience
\end{rSubsection}

\end{rSection}


%----------------------------------------------------------------------------------------
%    PROJECTS
%----------------------------------------------------------------------------------------
\begin{rSection}{\large Projects}

 \begin{rSubsection}{Huffman Compression Algorithm}{}{}{}
\item Implemented Huffman Compression and Decompression in C
\item Created fundamental data structures including nodes, priority queues and stacks 
\item Used these fundamental data structures in tandem
\end{rSubsection}

\begin{rSubsection}{Pintos Project}{}{}{}
\item Changed the way Pintos puts threads to sleep as to not use busy wait
\item Implemented a priority ready queue that takes into account processes' priority when scheduling onto the CPU
\item Implemented system calls such as exec, wait, etc
\end{rSubsection}

\begin{rSubsection}{Hamming Encoding}{}{}{}
\item Implemented the error correction Hamming encoding algorithm in C
\item Used data structures like bit vectors and bit matrices
\end{rSubsection}

 \begin{rSubsection}{Simple Motion Detection}{}{}{}
\item Implemented a simple motion detection algorithm
\item Created sliding window Averagerators 
\end{rSubsection}

\begin{rSubsection}{Faster Merge Sort}{}{}{}
\item Made Merge Sort run faster using UNIX and POSIX child processes
\item Also made a multi-threaded implementation using POSIX threads
\end{rSubsection}

\end{rSection}


%----------------------------------------------------------------------------------------
%	SKILLS SECTION
%----------------------------------------------------------------------------------------
\begin{rSection}{\large Skills}
\begin{rSubsection}{Programming Languages}{}{}{}
\item Python, Java, C, C{}\verb!++!, MIPS Assembly, RISC-V Assembly
\end{rSubsection}

\begin{rSubsection}{Technologies}{}{}{}
\item Git, Bash, LaTeX, ActionScript, BeautifulSoup
\end{rSubsection}

\end{rSection}



\end{document}
